\chapter{Quick start} % Write in your own chapter title
\label{Chapter2}
%\lhead{Chapter 1. \emph{Introduction}} % Write in your own chapter title to set the page header

This chapter intends to briefly get the reader up to speed with an example IM installation. There are three types of files that IM needs to find in its scripts directory: the configuration file, recipe files in \texttt{SCRIPTROOT/recipes} subdirectory, and action definition files, in \texttt{SCRIPTROOT/actions}. The location of the scripts directory, as well as the name of the configuration file, can be specified as command line arguments to the IM executable, \texttt{imcore}.


\section{Running a standalone process}

Command: \texttt{imcore}

\vskip 2mm

\noindent Command line options:

\begin{itemize}
\item[] \texttt{-h} \hspace{2cm} Prints command line options information

\item[] \texttt{-d level} \hspace{2cm} One of the following debug levels: \texttt{ERROR}, \texttt{WARNING}, \texttt{INFO}, \texttt{DEBUG}, \texttt{DEBUG1}, \texttt{DEBUG2}, \texttt{DEBUG3}, \texttt{DEBUG4}

\item[] \texttt{-t timer\_period} \hspace{2cm} If present, this option causes IM to run in timer mode, where ever timer\_period seconds IM state update is forced.

\item[] \texttt{-l path/log\_file\_prefix} \hspace{2cm} If present, this option causes all the debug output be directed to the file prefixed

\item[] \texttt{-x} \hspace{2cm} If present, this option enables the text-based status of IM displayed in the current terminal window.

\item[] \texttt{-p path} \hspace{2cm} If present, override the default path of IM scripts directory.

\item[] \texttt{-i init\_file} \hspace{2cm} If present, overrides the default init file name located in the root of IM scripts directory.

\end{itemize}

\section{Configuration file}

IM configuration file is located in the root of scripts directory, and by default is named \texttt{initialize\_im.xml}. The default init file name can be changed by \texttt{-i init\_file} command line option.

The initializer file has three sections: list of triggerable recipes, list of recipe files and list of action files---see the example below.

\lstset{language=XML}
\begin{lstlisting}
<?xml version="1.0" encoding="US-ASCII"?>
 
<iminit>
  <!-- these recepies are always ready to be triggered by im itself-->
  <triggerables>
     <recipe name="engage user name qa" max_instances="any"/>
     <recipe name="heartbeat" max_instances="1"/>
  </triggerables>
 
  <recipefiles>             
   <file>engageuser.xml</file>
   <file>engageuser_seq.xml</file>
   <file>engageuser_name_qa.xml</file>
   <file>answer_weather_followup.xml</file>
   <file>answer_weather_followup_yes.xml</file>
   <file>answer_weather_followup_no.xml</file>
   <file>heartbeat.xml</file>   
   <file>chatwithuser.xml</file>
  </recipefiles>

  <actionfiles>
    <file>defaults.xml</file>
    <file>say.xml</file>
    <file>misc.xml</file>
    <file>angry.xml</file>
    <file>common.xml</file>
  </actionfiles>
</iminit>
\end{lstlisting}

\section{Recipes}

IM is a production system, with rules specified as recipes. A recipe consists of the following elements:

\begin{itemize}
\item XML header: 
\begin{lstlisting}
  <?xml version="1.0" encoding="US-ASCII"?>
\end{lstlisting}

\item A unique recipe name: 
\begin{lstlisting}
  <recipe name="engage user name qa">
\end{lstlisting}

\item A precondition, which is either a Formula (a disjunction of a few conjunctions), or a set of Atoms corresponding to a single conjunction: 
\begin{lstlisting}
<!-- preconditions: unification amd bindings -->
<precondition>
    <atom quantifier="exist">
      <!-- object type and subtype -->
      <slot name="type" val="im"/>
      <slot name="subtype" val="user"/>
      <!-- arguments -->
      <slot name="present" val="true"/>
      <slot name="engaged" val="false"/>
      <slot name="has_been_engaged" val="false"/>
      <!-- bindings -->
      <slot name="invite_string" rel="bind" var="hala_invite_string"/>
      <slot name="id" rel="bind" var="present_user_id"/>
      <slot name="this" rel="bind" var="present_user_atom"/>
    </atom>
</precondition>
\end{lstlisting}

\item A whilecondition, which is either a Formula (a disjunction of a few conjunctions), or a set of Atoms corresponding to a single conjunction:
\begin{lstlisting}
<!-- purge the recipe when this condition fails -->
<whilecondition>
  <atom>
    <slot name="present" val="true"/>
    <slot name="this" var="present_user_atom"/>
  </atom>
</whilecondition>
\end{lstlisting}

\item A body of the recipe, which is a sequence of one of the following elements: an assignment, an action, a goal. By default the steps are performed sequentially and conditionally on the successful execution of the preceding steps. 

\begin{lstlisting}
<body order="sequence">
  <assignment>
          <atom>
              <slot name="engaged" val="true"/>
                <slot name="this" var="present_user_atom"/>
          </atom>
  </assignment>
  <action name="say_hello_ask_name" actor="robot" if_node_purged="abort"/>
  <action name="say_name" actor="user"/>
  <action name="say_nice_to_meet_you_name" actor="robot"/>
  <goal recipe_name="any" initiator="any">
    <atom>
      <slot name="uu_unhandled" val="false"/>
      <slot name="this" var="present_user_atom"/>
    </atom>
  </goal>
  <action name="say_goodbye_name" actor="robot"/>
 <!--  -->
</body>
\end{lstlisting}

\item An assignpost, which is an assignment that is to be performed upon the completion of the recipe. This assignment are equivalent to the assignments in the end of the recipe body, except for they are also used in matching the recipe against the currently active goal. 

\begin{lstlisting}
<!-- set right after execution ends -->
<assignpost>
  <atom>
    <!-- set object which name is equal to the one stored in var -->
    <slot name="this" var="present_user_atom"/>
    <slot name="has_been_engaged" val="true"/>
    <slot name="engaged" val="false"/>
  </atom>
</assignpost>
\end{lstlisting}

\section{Action definition files}

Actions are the outputs of the Interaction Manager. They consist of a list of arguments, sequence of datablocks, and elements defined by the Behavior Markup Language~\citep{BMLweb}.


Here is an action with an argument and two datablocks:
\begin{lstlisting}
<?xml version="1.0" encoding="US-ASCII"?>

<bml name="angry_response1">
    <roboml:args>
        <arg name="arg_utterance" type="string"/>
    </roboml:args>

    <roboml:datablock>
        <animation_id>frown1</animation_id>
        <intensity>1.0</intensity>
        <text>$arg_utterance</text>
        <focus></focus>
        <head></head>
    </roboml:datablock>

    <roboml:datablock>
        <animation_id>frown2</animation_id>
        <intensity>1.0</intensity>
        <text></text>
        <focus></focus>
        <head></head>
    </roboml:datablock>
</bml>
\end{lstlisting}




\end{itemize}
